\chapter{Conclusions}\label{conclusions}


In this PhD dissertation, I (along with other collaborators on different projects) tried
to build a better understanding of the distribution of 
matter in the Universe and its clustering properties, which I studied
in two very different regimes: gravitationally  bound systems like 
cluster of galaxies, and the large scale structure of the Universe. 
While working on various projects on this topic, I successfully
completed five scientific publications:

\begin{itemize}
\item	The very first project is a mass modelling problem in strong lensing cluster
	\cite{2014MNRAS.439.2651M}. Here we
	used a publicly available code GRALE, modified it for an optimum solution and resolution,
	to put tight constraints on the mass distribution of few lensing clusters. By studying
	the mass and light distribution of these clusters, we imply that it is possible 
	that the dark-matter has a finite self-interaction cross section and shows
	important signatures in the central parts of clusters. 

\item	In a similar project, we show that the time delay information is very useful in 
	order to put additional constraints on the central region of the clusters which can't 
	be resolved directly, especially when the steepness degeneracy is broken by the 
	presence of background sources at different redshift \cite{2015PASJ...67...21M}.

\item	We also presented a (semi) analytic model for the matter power spectrum, which is
	computationally inexpensive and computes the power spectrum to a percent level 
	accuracy up to k $\sim$ 1 $h^{-1}\mathrm{Mpc}$ \cite{2014MNRAS.445.3382M}. 
	The motivation of this model is the halo model
	and we also derived a simple form of the covariance matrix. We proposed a way to 
	marginalise over baryonic effects. 

\item	In another project, we used the halo model in order to directly
	model the effects of baryons on the matter power spectrum and to the extension on the
	weak lensing shear power spectrum \cite{2014arXiv1410.6826M}. 
	The effects are small at comparatively large
	scales but as the next generation surveys are expected to measure these quantities 
	to very small scales, the baryonic effects are important to take into account. If not, 
	it will add biases to the cosmological parameters up to 10 sigma. 

\item Finally, we produced free-form  mass models for six Hubble Frontier Field clusters.
	Their mass distribution shows elongation, multiple-cores, and many sub-structures
	indicating a recent major merger. We measured the power spectrum of the mass 
	distribution to quantify the sub-structures, and compared them to the simulated
	clusters.

\end{itemize}



%------------------------------------------------------------------------------


\subsection{A unified picture}

If we try to draw a bigger picture from all the projects above, 
it states that it is very important to model baryonic physics 
in order to understand the clustering processes,
galaxy formation, and to do cosmology with future generation surveys. 

Gravitational lensing is one ideal tool to do accomplish this. Strong lensing
is very useful in studying the individual systems. 
Weak lensing shear measurements
on larger area in the sky is the most promising tool to do cosmology under
control systematics. One of the biggest source of systematics is again the 
baryonic effects at small scales, and there are various ways to handle them. But
they can't be ignored. Precision cosmology and detailed information about the 
individual systems is needed in order to gain full understanding of the 
galaxy formation processes and evolution of the Universe. 








