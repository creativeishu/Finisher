\chapter{Introduction}\label{Introduction}

Our understanding of the Universe has advanced with the advancement in our 
theoretical, observational and computational abilities in the last two decades. 
While we are confident about the flatness of the Universe from CMB experiments, 
the indicative accelerating Universe from Supernovae-Ia surveys is well supported
by a cosmological constant. Finally a non-interacting dark matter is essential 
to explain the kinematics of galaxies in clusters, lightcurves of galaxies, 
gravitational lensing etc. A comprehensive model, the so-called $\Lambda$CDM,
is very successful in explaning most of the observations independently and 
combined constraints tells us that our Universe is composed of almost 70$\%$
dark-energy the form of cosmological constant, about 25$\%$ non-interacting
cold dark-matter and only 5$\%$ of the baryonic/ordinary matter. 

%------------------------------------------------------------------------------

\section{An Expanding Universe}

A stationary Universe would be boring, we have an expanding one - even better its
accelerating. The evidences are found in various probes extending to SNe-Ia \cite{}, 
CMB \cite{}, BAO \cite{}, galaxy clustering \cite{}, Lyman alpha \cite{} etc. 
The state-of-the-art was seeded when somebody sees something \cite{} which is 
further improved/informed by someone else \cite{}.

Adding the so-called {\it cosmological principle} to the Einstein's field equations, 
which relates the Einstein tensor (geometrical object) to the Energy-Momentum tensor, 
derives the Friedmann equations:

\begin{equation}
\centering
\left( \dfrac{\dot{a}}{a} \right)^2 = \dfrac{8 \pi G}{3} \rho - \dfrac{Kc^2}{a^2},
\label{eqn:fe1}
\end{equation}

\begin{equation}
\centering
\left( \dfrac{\ddot{a}}{a} \right) = -\dfrac{4 \pi G}{3} \left( \rho + \dfrac{3p}{c^2} \right),
\label{eqn:fe2}
\end{equation}
\\
where, $\rho$ and $p$ are the density and pressure of the corressponding component of
the energy density of the Universe, $a$ is the scale factor normalised to unity at present.
$K$ is the curvature parameter and follows $K=0$ for a flat Universe. $G$ and $c$ are
the Gravitation constant and speed of light respectively. Related by equation \ref{eqn:fe1} 
and \ref{eqn:fe2}, the dynamics and expansion of the Universe depend on its contents. 
Combining these two equations give a third equation, also known as {\it thirt} Friedmann
equation:

\begin{equation}
\centering
\dot{\rho} = -3 \dfrac{\dot{a}}{a}(\rho + p).
\label{eqn:fe3}
\end{equation}

These three equations completely describe the dynamics of the Universe at all times.
The first two equations relates the matter variables ($\rho$ and $p$) to the geometric
variables ($a$ and $K$) whereas the third equation resembles the first law of 
thermodynamics for adiabatically expanding Universe, which is a fair approximation.

One can also relate $\rho$ and $p$ with an {\it equation of state}:

\begin{equation}
\centering
p = w \rho,
\label{eqn:eos}
\end{equation}
\\
$w$ is known as the equation of state variable for the respective component
of the energy density of Universe. For non-relativistic matter like dust $w=0$,
for photons and relativistic neutrinos $w=1/3$ whereas for a cosmological constant
$w=-1$.




%------------------------------------------------------------------------------

\section{Structure Formation in the Universe}

Starting from CMB and Recombination era, talk about homogeneous Universe, matter-
dominated Universe, dark-ages etc.

Hierarchical structures, observational constraints, Re-ionization.

%----------------------------------------------
\subsection{Linear theory}

Derivation of growth equation, with minimal equations of physics, particularly
continuity equation, Euler equation and Poisson equation and reach to this:

\begin{equation}
\centering
\dfrac{\partial^2\delta}{\partial t^2} + \dfrac{2\dot{a}}{a} \dfrac{\partial\delta}{\partial t}
 - \dfrac{3H_0^2 \Omega_m}{2a^3} \delta = 0
\end{equation}

Random fields, power spectrum and two point statistics.
%----------------------------------------------
\subsection{Non-Linear theory}
\subsubsection{Spherical collapse model}
\subsubsection{The halo model}
\subsubsection{N-body simualtions}
\subsubsection{Dark-matter models}
\subsubsection{Covariance matrix of matter power spectrum}
%----------------------------------------------
\subsection{Baryonic contributions}
%----------------------------------------------
\section{Probes of Cosmology}

Starting general ideas and all possible probes of cosmology with preferences.

%------------------------------------------------------------------------------

\subsection{Gravitational Lensing}
%----------------------------------------------
\subsubsection{Strong Gravitational Lensing}
%----------------------------------------------
\subsubsection{Weak Gravitational Lensing}
%----------------------------------------------
\subsubsection{Other Lensing Observables}

\section{Statistics}
\subsection{Bayesian world}
\subsection{MCMC and Fisher}
\subsection{Genetic algorithm}

\section{Motivation}

Here I would like to give a comprehensive summary of this work.

\subsection{Challenges}
\subsection{What are we trying to achieve}
\subsection{A unified picture}



\clearpage
\cite{2014MNRAS.440.2290M}

\cite{2014MNRAS.439.2651M}

\cite{2014A&A...567A..65B}

\cite{2014MNRAS.445.3382M}

\cite{2014arXiv1410.6826M}

\cite{2015PASJ...67...21M}

\cite{2015arXiv150403388M}





