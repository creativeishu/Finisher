\chapter{Paper3: Baryonic effects on weak-lensing two-point statistic}\label{paper:baryoniceffects}

According to the Dark Energy Task Force (DETF), weak lensing is amongst the most promising
tools to do cosmology and derive essential constraints on cosmological parameters. 
However, it is dominated by many systematics both observationally and theoretically.
In this paper, we focussed on one very important source of systematic errors in the 
theory of weak lensing; {\it baryonic effects}. 

Modeling the two-point statistics or the power spectrum of the weak-lensing shear measurements
is described in section \ref{} for assuming a dark-matter only Universe. Which is a fair approximation
at large scales and was the former limit to the weak-lensing experiments. However, the next
generation surveys like Euclid, LSST etc, are pushing this limit so far to the small scales
that baryonic contribution becomes important. In this paper, we provide a model
to incorporate baryonic contribution in the modeling of the matter power spectrum
and to the extension weak lensing shear power spectrum. Our model is based on 
the halo model where the baryonic contribution is sensitive to two quantities: 
First, the halo mass function and second, the radial density profiles of the haloes.

Our baryonic model consists of hot intra cluster gas in hydrostatic equilibrium, 
stellar component dominated by a central galaxy which is constrained by the abundance
matching techniques, An feedback model that removes gas from the haloes as a function
of their mass and an adiabatically contracted dark-matter component. In corporating
these component, we can reproduce many different hydrodynamical simulations both
for the radial profile of the haloes and the matter power spectrum. Our model has only
one free parameter, all the rest are constrained by observations or simulations. 

Incorporating this baryonic model in the pipeline to compute the shear power spectrum, 
we did a cosmological parameter forecasts for a Euclid like survey and found that using
weak-lensing alone, Euclid is expected to constrain the cosmological parameters to very
high accuracy, however, if the baryonic effects are not taken into account, it will bias
the recovered values of the parameters and mislead the interpretations. On the other hand,
if the baryonic effects are incorporated, at least in this kind of model, all cosmological
parameter can still be constrained with good accuracy along with the baryonic effect
parameter. 




\clearpage
\includepdf[pages=-]{/Users/Irshad/Dropbox/mypapers/baryoniceffects_irshad_davide_romain_adam.pdf}