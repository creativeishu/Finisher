\chapter{Paper3: Baryonic effects on weak-lensing two-point statistics}\label{paper:baryoniceffects}

According to the Dark Energy Task Force (DETF; \cite{2006astro.ph..9591A}), 
the two-point statistics of the weak gravitational lensing is amongst the most promising
tools to do cosmology and derive strong constraints on the cosmological parameters. 
However, it is dominated by many systematics, both observational and theoretical.
In this paper, we focussed on an extremely important source of systematic errors in the 
theory of weak lensing shear power spectrum - {\it baryonic effects}. 

The theoretical modelling of the weak-lensing shear power spectrum
is described in section \ref{sec:wl} for a dark-matter only Universe, 
which is a fair approximation
at large scales, which was the limit to the weak-lensing experiments in the past. 
The next generation surveys like Euclid, LSST etc, 
are pushing this limit far into the small scales such
that baryonic contribution becomes important. In this paper, we built a model
to incorporate baryonic contribution in the matter power spectrum,
and by extension to the weak lensing shear power spectrum. Our model is based on 
the halo model where the baryonic contribution is sensitive to mainly two quantities: 
the halo mass function, and the radial density profiles of the halos.

Our baryonic model consists of four main ingredients: 
(i) hot intra cluster gas assumed to be in hydrostatic equilibrium;
(ii) a stellar component dominated by a central galaxy 
whose mass is constrained by the abundance matching techniques;
(iii) a feedback model that removes the gas from the halo as a function
of its mass; and 
(iv) an adiabatically contracted dark-matter component. 
Incorporating these four components, 
we can reproduce hydrodynamical simulations 
for the radial profile of the halos, and the matter power spectrum. 
%This baryonic model has only
%one free parameter , all the rest are constrained by observations or simulations. 

We performed a cosmological parameter forecast for a Euclid like survey and found that using
weak-lensing alone, Euclid is expected to constrain the cosmological parameters to a very
high accuracy. However, if the baryonic effects are not taken into account, it will bias
the recovered values of the parameters and mislead the interpretations. On the other hand,
if the baryonic effects are taken into account, all cosmological
parameters can still be constrained with good accuracy along with the parameters
of the baryonic model.

This paper has been submitted to {\it Monthly Notices of the Royal Astronomical
Society} (MNRAS), and is currently under review.


\clearpage
\includepdf[pages=-]{/Users/Irshad/Dropbox/mypapers/baryoniceffects_irshad_davide_romain_adam.pdf}