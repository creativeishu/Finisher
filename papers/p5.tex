\chapter{Paper5: Quantifying sub-structures in lensing clusters}\label{paper:substructures}

Numerical simulations has provided an insight to the theory of structure formation. 
Large collapsed objects, like cluster of galaxies, form by a series of mergers of 
small mass systems, like galaxies or group of galaxies, and associated gas, dark-matter
halos etc. The abundance of sub-structures in a merging system can be quantified 
to study the merging stage of the cluster. This evaluation is trivial using simulations. 
The observed macroscopic properties of galaxies and galaxy clusters, like gas mass, bulge sizes
etc., can be well reproduced using the current generation hydrodynamical simulations. 
However, a more ambitious comparison would be the total mass distribution in the
clusters. As the total mass distribution in an ongoing merger is the result
of an initial random field evolved with deterministic forces, only statistical
properties can be derived, and not the actual distribution. The derived statistical
quantities like radial density profile, sub-structure mass function etc., can be
compared to that of the observed clusters to validate the simulation
ingredients. 

Tracing the mass distribution in observed clusters is a non-trivial task, as 
mass is not an observable, its light. Gravitational lensing, particularly
strong lensing, is an unbiased and
ideal tool to estimate the total mass distribution. However, the resolution of the
reconstructed mass maps depends on the priors and the lensing data. In order to
have unbiased estimates, priors must be minimal, and the total mass distribution
must be inferred using lensing data alone. Therefore, we need non-parametric
mass reconstruction methods without light-traces-mass assumption, and higher
density of lensed images of the background sources, in order to correctly model 
the mass distribution of the lens (or the cluster).

Such large quantity of lensing data has recently been provided by Hubble Space
Telescope (HST) under Hubble Frontier Field (HFF) discretionary program. HFF 
consists of six massive cluster of galaxies in redshift range 0.3 to 0.5. The HFF
lensing dataset contains nearly 200 lensed images in each cluster, along 
with the good estimates of their redshift. 

In this paper, we produced mass models for the six HFF clusters using pre-HFF
data, and for one amongst them using HFF data. We report the gain in the
spatial resolution of mass maps using HFF data over pre-HFF data. To make
these mass maps, we use GRALE, a strong gravitational lensing non-parametric 
mass reconstruction technique, without assuming any light information from
the clusters, except for their redshift. The mass distribution of all HFF
clusters show elongation, multiple-cores, and many sub-structures, 
indicating a recent major merger. Therefore, extracting radial profiles
of the clusters is not very encouraging. Also, because gravitational lensing
can only estimate the sky-projected 2D mass distribution, many sub-structures
are erased, and an estimate of the sub-structure mass function
cannot be derived without LTM assumption. 

We proposed the power spectrum of the 2D mass distribution as an estimator
for the sub-structure. We measured this quantity for all HFF clusters, and 
found large power at small scales for clusters at low redshift, or with
larger number of lensed images. We further made similar measurements of the
power spectrum from simulated clusters, both dark-matter only and 
hydrodynamical, and
found that the average power in simulated clusters is larger than that
of the observed clusters at small scales. We discussed the possible reasons
for this could be: (i) limited lensing data, (ii) contrast in redshift between
observed clusters and simulations, or (iii) lack of physics in simulations.


{\bf Role:} For this project, I used the mass maps made by Professor
Liliya. L. R. Williams and Kevin Sebesta. 
I proposed the idea to measure the 2D power spectrum
of the mass distribution. I started
by doing a null test, which shows equal power at all scales if there is
only noise in the field. I then generated three random cluster fields 
for a given power spectrum using a code written by Dr. Prasenjit Saha. I 
measured the power spectrum of these simulated fields, and compared it to 
the original power spectrum; a good agreement was found. I accumulated 
simulated clusters data from Davide martizzi, and measured the power spectrums
of all 66 clusters, each for DMO and hydrodynamical simulations. Finally 
I made average power spectrums of simulated and observed clusters, and
draw a comparison.









This paper has been submitted to {\it Monthly Notices of the Royal Astronomical
Society} (MNRAS), and is currently under review.
\\
Arxiv: \url{}

\clearpage
\includepdf[pages=-]{/Users/Irshad/github/HubbleFrontierField/paper1_substructures/ms.pdf}