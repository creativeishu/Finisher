\chapter{Paper1: Mass–galaxy offsets in lensing clusters}\label{paper:massgalaxyoffsets}

In the concordance model of cosmology, cold dark matter (CDM) provides the major
budget for the matter content in the Universe, nearly 80$\%$. 
It interacts only gravitationally and provides a large potential well that
attracts low mass halos to form high mass halos
through a series of mergers. Baryons follow these potential wells
and cool down to form stars and galaxies. Therefore, it is safe to assume
that light-follows-mass in these giant halos. However, there could be a number of
astrophysical and/or cosmological scenarios where light-follows-mass does not hold. 
For example, if the dark matter interacts with baryons or with itself. This
kind of non-standard properties of the dark-matter can be tested in 
dense regions like the centre of the galaxy clusters. Strong gravitational lensing is 
an ideal tool to obtain this information as it is sensitive to the total
mass of the lens, and does not differentiate between dark-matter or luminous
galaxies.

In this paper, we studied the mass-galaxy offsets in three lensing clusters of galaxies:
Abell 3827, Abell 2218 and Abell 1689. These three clusters are very different in 
their morphology, total mass, redshift, and lensing data. We used GRALE, a strong
gravitational lens inversion library, to model the mass maps of these clusters given
the position of the multiple images of the background sources 
and their redshift. No information
from the lensing clusters was used, except for their redshift. The
mass models are completely form free. We also provide the uncertainty
maps that show high-confidence in the region where lensing images were present. 

Because of the free-form and high certainities of the mass maps, 
it is possible to compare the distribution
of matter in these clusters with luminous galaxies
completely on the basis of statistical uncertainties. In Abell 3827 and
Abell 2218 we found small offsets between the local mass peaks and the
position of a nearby galaxy. Particularly, in Abell 3827 this offset is nearly 6 kpc,
and is statistically significant. We also discussed the possible origin 
of these offsets, which can be either astrophysical like dynamical
friction or the non-standard properties of the dark-matter component like
self-interactions. 
The offset in Abell 3827 is further studied by Massey et al. 2015 with new HST
data where the offset was found to be robust. With a simplified model, we argued that
to explain this offset, the cross-section of dark-matter particles 
$\sigma/m$ must be of the order $10^{-4}$.

In Abell 1689, no significant offsets were found. However,  
we found a line of sight sub-structure behind the cluster at
redshift $\sim 3$. 

This paper has been published in {\it Monthly Notices of the Royal Astronomical
Society} (MNRAS). 



\clearpage
\includepdf[pages=-]{/Users/Irshad/Dropbox/mypapers/massgalaxyoffset_irshad_jori_prasenjit_liliya.pdf}