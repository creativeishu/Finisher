\chapter{Paper1: Mass–galaxy offsets in lensing clusters}\label{paper:massgalaxyoffsets}

In the concordance model of cosmology, cold dark matter (CDM) provides the major
budget of matter density in the Universe, nearly 80$\%$ and only 20$\%$ of baryons. 
The CDM that interacts only gravitationally provides large potential well and 
attracts many small mass systems and form giant galaxy clusters or large
mass haloes through a series of mergers. Baryons follows these potential wells
and cools down to form stars or galaxies. Therefore, it is a safe assumption
that light follows mass in these clusters. However, there could be a number of
astrophysical and cosmological scenarios where mass follows light does not hold. 
For example, if the dark matter interacts with baryons or with itself. This
kind of non-standard properties cannot be tested unless the environment is dense,
like close to the centre of galaxy clusters. Strong gravitational lensing is 
an ideal tool to obtain this information as it is sensitive to the total
mass of the lens and do not differentiate between dark-matter or luminous
galaxies. 

In this paper, we studied the mass-galaxy offsets in three lensing clusters of galaxies:
Abell 3827, Abell 2218 and Abell 1689. These three clusters are very different in 
their morphology, total mass, redshift and lensing data. We used GRALE, a strong
gravitational lens inversion library, to model the mass map of these clusters given
the multiple images of the background sources and their redshift. No information
from these clusters were used in this modelling, except for their redshift. The
mass models are completely form free and unbiased. We also provided the uncertainty
maps that shows high-confidence in the region where lensing images were present. 


Because of the free-form and high certainities of the mass models, 
it is possible to compare the distribution
of matter in these clusters with luminous galaxies without any biases 
completely on the basis of statistical uncertainties. In Abell 3827 and
Abell 2218 we found small offsets between the local mass peaks and the
position of nearby galaxy. Particularly, in Abell 3827 this offset is nearly 6kpc
and statistically significant. We also discussed the possible origin 
of these offsets, which can be completely astrophysical like dynamical
friction or may be the non-standard properties of the dark-matter component like
self-interactions. 

In Abell 1689, there is no significant offset was found. However, with a careful
modeling, we found a line of sight sub-structure behind the cluster nearly at
redshift of 3. 

The offset in Abell 3827 is further studied by Massey et al. 2015 with new HST
data and these offsets found to be robust. With a simplified model, we found that
to explain these kind of offsets, the cross-section of the darkmatter particles 
$\sigma/m$ must be of the order $10^{-4}$.


\clearpage
\includepdf[pages=-]{/Users/Irshad/Dropbox/mypapers/massgalaxyoffset_irshad_jori_prasenjit_liliya.pdf}