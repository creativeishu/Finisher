\chapter{Paper4: Lensing time delays as a substructure constraint}\label{paper:timedelays}

Strong gravitational lensing (SL) is a very powerful tool to reconstruct the mass distribution
of the lens given the positions of the multiple images of the background source(s). 
Another observable
of the SL phenomenon is the time delay between two images of the same source,
due to the fact that two light rays from the same source, emitted at the
same time, travel different paths (and so the distances), and reach the same observer. 
If the source is variable, like a quasar or a supernovae etc., 
it is possible to measure these time delays. 

As described by equation \ref{eqn:timedelay}, 
the total time delay between two images 
is proportional to the comoving distance from the observer to the lens, 
and therefore depends
on the Hubble constant $H_0$. If the mass distribution of the lens
is known, it is possible to infer the $H_0$ by measuring the time delays.
However, this exercise is very sensitive to the mass distribution of the 
lens. Therefore, a very good understanding of the gravitational potential
of the lens is needed in order to measure the $H_0$ using the time delays.
This has been done by many authors in the past, where they derived necessary
constraints on $H_0$.

In this paper, we quantified the additional constraints provided by the time delays
on the mass distribution of the lens assuming a cosmological model (or $H_0$). 
We performed a principle component analysis (PCA)
using two types of mass maps of the lensing cluster SDSS J1004+4112 - one 
reconstructed with time delays data (TD) and other without it (NTD). The main
science driver of this paper is to identify and isolate those 
uncertainty modes that are present in NTD maps and are not in TD maps.
By successfully identifying these modes, we concluded that in the lensing clusters
where the steepness degeneracy is already broken by multiple background sources
at different redshifts, time delay information can be used to constrain
the lopsidedness of the cluster core.

{\bf Role:} Under the supervision of Dr. Prasenjit Saha, I started this project
by accumulating lensing data for SDSS J1004 cluster. Along with the multiple
images of three background sources, two time delays were also available for one 
of the background sources, which is a quasar. Employing this dataset, I reconstructured
30 mass maps using lensed images only, and 30 using lensed images plus two time
delays. I also computed the average mass map with optimal resolution in both
cases. The next step was to isolate the uncertainty modes that were present
in NTD maps, and not in TD maps. Each mass map is a grid of pixels, and for each
case total 30 maps were available. 
I computed the uncertainty/moment matrix for NTD models.
I isolated its principle component, which was the eigen-vector corressponding
to the largest eigen-value, that resembles the largest uncertainty
mode present in NTD models, and compared its effect to the reference 
mass model that we chose to be the average TD mass map. 

This paper has been published in {\it Publication of Astronomical Society of Japan} (PASJ). 
\\
Arxiv: \url{http://arxiv.org/abs/1412.3464}

\clearpage
\includepdf[pages=-]{/Users/Irshad/Dropbox/mypapers/timedelaysJ1004_irshad_prasenjit_jori.pdf}