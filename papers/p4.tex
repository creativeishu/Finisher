\chapter{Paper4: Lensing time delays as a substructure constraint}\label{paper:timedelays}

Strong gravitational lensing is a very powerful tool to reconstruct the mass distribution
of the lens given the multiple images of the background source(s). Another observable
of the strong lensing phenomenon is the time delays between two images of the same source
which is due to the fact that the two light rays from the same source emitted at the
same time travel different path and reaches the same observer. If the source is variable,
like a quasar or a supernovae etc, it is possible to measure these time delays. 

In this paper, we quantified the additional constraints provided by the time delays
on the mass distribution of the lens. We performed a principle component analysis (PCA)
using two kind of reconstructions of the lensing cluster SDSS J1004+4112 - one with
time delays and other without time delays. 

\clearpage
\includepdf[pages=-]{/Users/Irshad/Dropbox/mypapers/timedelaysJ1004_irshad_prasenjit_jori.pdf}