\chapter{Paper2: Analytic model for the matter power spectrum}\label{paper:analyticmodel}

The future generation cosmological surveys, like Euclid, 
LSST etc., are expected to provide
very large quantity of high quality data such that it will be 
possible to probe small scale structures like never before. Employing 
the cosmological observables from these experiments,
like weak gravitational lensing, baryon accoustic oscillations etc., 
percent level constraints are expected on all cosmological 
parameters. With such tight
constraints, either the $\Lambda$CDM model will gain more credibility
in the community else it would be rejected with higher confidence; either 
way the efforts will be unified. 
In order to employ the full constraining power from these experiments, a good
understanding of non-linear structure formation is needed.
In this paper, we studied an important aspect of this theory -- 
the matter power spectrum. It underlies
many cosmological observables, it is important
to model it accurately and precisely up to the non-linear regime. 

In this paper, we provide an estimator for the matter
power spectrum based on the Zeldovich approximation and the halo model. 
This model is calibrated on $N-$body simulations, and gives
an accuracy of a few percent up to $k\sim 0.8 h^{-1}\mathrm{Mpc}$ 
over a range of cosmological models 
including neutrino masses and redshifts. 

We also provide an estimator for the full covariance matrix of the matter power
spectrum which is very important for statistical inference from the cosmological
data. In spite of the simple form of the covariance estimator, it is found
to be in remarkable agreement with simulations. 

We provide a description of baryonic effects on the matter
power spectrum in this framework. This model can be used to project and estimate
weak lensing power spectrum and utilise the future generation surveys
to put strong and unbiased constraints on the cosmological parameters.


\clearpage
\includepdf[pages=-]{/Users/Irshad/Dropbox/mypapers/analyticmodel_irshad_uros.pdf}