\chapter{Paper2: Analytic model for the matter power spectrum}\label{paper:analyticmodel}

The future generation cosmological surveys, like Euclid, 
LSST etc., are expected to provide
very large quantity of high quality data such that it will be 
possible to probe small scale structures like never before. Employing 
the cosmological observables from these experiments,
like weak gravitational lensing, baryon acoustic oscillations etc., 
percent level constraints are expected on all cosmological 
parameters. With such tight
constraints, either the $\Lambda$CDM model will gain more credibility
in the community else it would be rejected with higher confidence; either 
way the efforts will be unified. 
In order to employ the full constraining power from these experiments, a good
understanding of non-linear structure formation is needed.
In this paper, we studied an important aspect of this theory -- 
the matter power spectrum. It underlies
many cosmological observables, it is important
to model it accurately and precisely up to the non-linear regime. 

In this paper, we provide an estimator for the matter
power spectrum based on the Zeldovich approximation and the halo model. 
This model is calibrated on $N-$body simulations, and gives
an accuracy of a few percent up to $k\sim 0.8 h^{-1}\mathrm{Mpc}$ 
over a range of cosmological models 
including neutrino masses and redshifts. 

We also provide an estimator for the full covariance matrix of the matter power
spectrum which is very important for statistical inference from the cosmological
data. In spite of the simple form of the covariance estimator, it is found
to be in remarkable agreement with simulations. 

We provide a description of baryonic effects on the matter
power spectrum in this framework. This model can be used to project and estimate
weak lensing power spectrum and utilise the future generation surveys
to put strong and unbiased constraints on the cosmological parameters.

{\bf Role:} This project was done under the supervision of Professor Uros Seljak. 
The starting point was analytical calculations using the halo model. The
idea was to expand the 1-halo term in Taylor series, and analytically
evaluate the coefficients. I accumulated 38 matter power spectra
from the cosmic emulator, each for its original cosmological node. I
also evaluated Zeldovich power spectra using code from Zvonimir Vlah
for the same 38 cosmological models and three redshifts (0.0,0.5,1.0).
I fitted the function $A_0 - A_2k^2 + A_4k^4$ to the difference between 
the full matter power spectra and Zeldovich
approximation, and compared
the fitted coefficients with those evaluated analytically from the halo model. 
We found big differences in $A_2$ and $A_4$ coefficients, whereas
$A_0$ remains the same. This shows the divergence of the halo model
from the true matter power spectrum in non-linear scales. 
I found very strong correlation between fitted coefficients and $\sigma_8$,
and therefore we fitted them using a single power law for all three redshifts
and computed the corresponding residuals. 
Further I fitted the residuals for its correlation to the 
effective slope $n_{\rm eff}$.
We provide these fitting functions in the paper to estimate the 
coefficients of the 1-halo term using the cosmological models
and redshifts. Given these fitting functions, we computed the matter
power spectra, and compared them to the original emulator output. 
I found an agreement of about a percent up to $k\sim 0.7 h^{-1}\mathrm{Mpc}$.
I accumulated the power spectra from van Daalen et al. 2011, for both
dark-matter only as well as hydrodynamical simulations, 
and to their difference, I fitted the similar function. 
The only alteration was that this time I 
recovered the change in the coefficients due to the baryonic effects. 
We found that $A_0$ is indifferent to the baryonic effects, while
$A_2$ and $A_4$ change significantly, implying the conservation of
mass inside the halo where the profile is changing due to baryons. 
In the the same vein, we argued that at scales where only Zeldovich
term is important, the covariance is dominated by the cosmic variance.
To compute the total covariance, we can compute the variance of 
each of the coefficients. However, due to the baryonic effects contaminating
$A_2$ and $A_4$, it is good to add the variance of $A_0$ only to the
total covariance, and marginalise over the other two parameters. 
This form of the covariance matrix is found to be in remarkable
agreement with the simulations. 

This paper has been published in {\it Monthly Notices of the Royal Astronomical
Society} (MNRAS). 
\\
Arxiv: \url{http://arxiv.org/abs/1407.0060}
\clearpage
\includepdf[pages=-]{/Users/Irshad/Dropbox/mypapers/analyticmodel_irshad_uros.pdf}