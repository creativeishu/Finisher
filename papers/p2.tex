\chapter{Paper2: Analytic model for the matter power spectrum}\label{paper:analyticmodel}

The future generation cosmological surveys, like Euclid, LSST etc, are expected to provide
very high quality and quantity of data that it will be possible to probe small scale structures
that was not possible before. Employing the cosmological observables from these experiment, 
percent level constraints are expected on all cosmological parameters. With such tight
constraints, either the so-called $\Lambda$CDM model will gain more credibility
from all over the community or it could be rejected with high confidence; either the 
way the efforts will be unified. 

However, there are many theoretical challenges on the way to precision cosmology. In this
paper, we studied on important aspect -- the matter power spectrum. As it underlies
many cosmological observables like weak lensing statistics, BAO etc, it is important
to model it accurately and precisely. Here, we provide an estimator for the matter
power spectrum, analytic in nature so computationally inexpensive, based 
on a modified version of the halo model -- so physically motivated. We calibrated
this estimator on  the Cosmic Emulator, which claim to be percent 
level accurate up to $k\sim 1 h/Mpc$, our estimator shares the same precision
over a range of cosmologies including neutrino masses and redshifts. 

We also provide an estimator for the covariance matrix of the matter power
spectrum which is very important for statistical inference from the cosmological
data. Finally, we provide a description of baryonic effects on the matter
power spectrum in this framework. This model can be used to project and estimate
weak lensing power spectrum and utilise the future generation survey
to put strong and unbiased constraints on cosmological parameters.


\clearpage
\includepdf[pages=-]{/Users/Irshad/Dropbox/mypapers/analyticmodel_irshad_uros.pdf}