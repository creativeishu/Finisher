\phantomsection
\addcontentsline{toc}{chapter}{Zusammenfassung}

%\begin{abstractslong}    %uncommenting this line, gives a different abstract heading
\chapter*{Zusammenfassung}%\begin{zusammenfassung}        %this creates the heading for the abstract page

Verteilung der Materie im Universum enthält Fülle von Informationen über
der Energieinhalt des Universums, ihre Eigenschaften und Evolution. Es kann
in zwei sehr unterschiedliche Regime untersucht werden. Zunächst wird im einzelnen durch Gravitation
gebundenen Systemen wie Galaxien, Galaxienhaufen etc .; zweiten, in der
Groß Strukturen des Universums. Jedes dieser Regime haben spezifische
Anwendungen und das Verständnis der Theorie der kollektiv verbessert
Strukturbildung und der Kosmologie. Zum einen Cluster von Galaxien
die größten gravitativ gebundenen Strukturen im Universum und besteht
der hunderte von Galaxien und Intra-Cluster Gas bewegt sich in der Potentialmulde
der große dunkle Materie Komponente. Von Vorteil der großen Potentialtopf, hoch
Dichte und einer hohen Temperatur des Gases kann mit Sonden untersucht werden, wie
Gravitationslinsen, Röntgenbeobachtungen etc. und bietet kosmischen
Laboratorien, die Wechselwirkungen der Baryonen und dunkler Materie oder studieren
Nicht-Standard-Eigenschaften der dunklen Materie, wenn überhaupt.
Zweitens sind die Großstrukturen des Universums aufgrund der Entwicklung gebildet
von winzigen Störungen im Bereich anfängliche Dichte über Gravitations
Instabilität und viele baryonische Prozesse. Deshalb kann durch die Untersuchung der
Verteilung der Materie im LSS ist es möglich, den anfänglichen zu beschränken
Bedingungen und / oder der kosmologischen Parameter.

In dieser Doktorarbeit, auf diese beiden Aspekte und erfolgreich fokussiert I
Schluss fünf wissenschaftliche Arbeiten, die in diesem Manuskript gebunden sind:

Zunächst untersuchten wir die Verteilung der Materie in drei Galaxienhaufen
und rekonstruiert ihre Massenverteilung mit nicht-parametrische Verfahren in
starken Gravitationslinseneffekt. In einigen Clustern, fanden wir signifikante Offset
zwischen den Dichte Gipfel und die nahen Galaxien. Wir argumentierten, diese Offsets
könnte astrophysikalischen Ursprungs oder einen Hinweis auf Selbst Wechselwirkungen
der Dunkelmaterie-Teilchen.

Zweitens, in einem anderen Projekt zu starken Gravitationslinsen Zusammenhang haben wir untersucht
die Wirkung der Zeitverzögerung zwischen den verschiedenen Bildern von der gleichen Quelle,
auf der Massenverteilung der Linsen Clustern. Wir fanden, dass in Clustern
wo die Steilheit Entartung wird bereits von mehreren Hintergrundquellen gebrochen
bei verschiedenen Rotverschiebungen können Zeitverzögerung Informationen werden verwendet, um zu beschränken
die Einseitigkeit des Clusterkerns.

Drittens, wir bauen ein analytisches Modell für die Sache Leistungsspektrum,
beschreibt die Materiedichte Schwankungen statistisch (nur zum zweiten
Reihenfolge). Das Modell ist rechnerisch kostengünstig und prognostiziert die
Egal Leistungsspektrum auf einen Prozent-Niveau Genauigkeit bis zu
$ k\sim h^{-1} \mathrm{Mpc} $.

Viertens haben wir untersucht die Auswirkungen der Baryonen auf den Himmel projiziert schwachen Linsen
Scherleistungsspektrum. Wir argumentieren, dass diese Effekte signifikant werden
auf kleinen Skalen $\ell \sim$ 5000 und bei Nichtbeachtung wird es spannen die
Interpretation der kosmologischen Parameter zu vielen Sigma.

Fünftens rekonstruierten wir die Massen Karten von sechs Hubble Frontier Field Clustern.
Ihre Massenverteilung zeigt Dehnung, multiple-Kerne und viele Unterkonstruktionen
Anzeigen einer letzten größeren Fusion. Wir quantifiziert auch deren Clustering
Grundstück mit dem Leistungsspektrum der Massen Feld und verglichen sie
mit $\Lambda$CDM simuliert Clustern.
% ---------------------------------------------------------------------- 
