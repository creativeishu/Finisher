\phantomsection
\addcontentsline{toc}{chapter}{Zusammenfassung}

%\begin{abstractslong}    %uncommenting this line, gives a different abstract heading
\chapter*{Zusammenfassung}%\begin{zusammenfassung}        %this creates the heading for the abstract page



Die Verteilung der Materie im Universum enth{\"a}lt eine F{\"u}lle von
Informationen {\"u}ber die Eigentschaften, den Energieinhalt und die zeitliche
Entwicklung des Universums. Sie kann in zwei sehr unterschiedlichen
Gr{\"o}ssenordnungen untersucht werden, einerseits in gravitativ gebundenen
Systemen oder Subsystemen wie zum Beispiel Galaxien und Galaxienhaufen und andererseits in
den gr{\"o}ssten  Strukturen (Large Scale Structures, LSS) des Universums.
In jedem dieser beiden Regime finden man spezifische Anwendungen und zusammen bilden sie ein Werkzeug f\"ur das Verst{\"a}ndnis der Strukturbildung und der Kosmologie im Allgemeinen.
Galaxienhaufen sind die gr{\"o}ssten gravitativ gebundenen Strukturen im
Universum und bestehen von Hunderten von Galaxien. Das Gas im Inneren bewegt
sich in der Potentialtopf welche von der dunklen Materiekomponenten des Systems
erzeug wird. Wegen der Tiefe des Potentialtopfs und der hohen Gasdichte und
Temperatur kann ein solcher Haufen mit Techniken wie Gravitationslinsen,
R{\"o}ntgenbeobachtungen usw untersucht werden.  Es bietet eine Art kosmisches
Labor, in dem die Wechselwirkungen zwischen Baryonen und dunkler Materie zu
studiert oder Abweichungen der dunklen Materie vom Standartmodell untersucht werden k\"onnen.
Die Gr{\"o}ssstrukturen andererseits entstehen aus dem gravitativen Kollaps winziger Fluktuationen im urspr{\"u}nglichen Dichtefeld. Das Studium der
Materieverteilung in LSS erm{\"o}glicht es, die Anfangsbedingungen und/oder
kosmologischen Parameter einzuschr{\"a}nken.


Im Rahmen dieser Doktorarbeit habe ich mich mit diesen beiden Aspekten
auseinandergestetzt und erfolgreich f{\"u}nf wissenschaftliche Arbeiten
publiziert welche in dieser Doktorarbeit in gebundener Form vorliegen.

Als erstes untersuchten wir die Verteilung der Materie in drei Galaxienhaufen
und rekonstruierten ihre Massenverteilungen mit Hilfe einer parameterfreien
Formulierung des starken Gravitationslinseneffekts. In zwei dieser Haufen fanden
wir signifikante Abweichungen zwischen den Dichtemaxima der Messungen und der
n{\"a}chsten Galaxie. Wir untersuchten, ob diese Abweichungen einen
astrophysikalischen Ursprung haben oder Hinweise auf Selbstinteraktionen
zwischen dunkler Materie sind.

Als n\"achstes haben wir --- einer {\"a}hnlichen Richtung folgend --- die
Auswirkungen der Massenverteilung des beobachteten Haufens auf die
Zeitverz\"ogerung zwischen den verschiedenen Bildern der selben Quelle untersucht.
Wir fanden, dass in Galaxienhaufen die Zeitverz{\"o}gerung zwischen Bildern
Informationen liefert, welche verwendet werden k{\"o}nnen um die Kugelsymmetrie
des Kerns des Haufens zu beschr{\"a}nken.

In den anderen Arbeiten haben wir ein analytisches Modell f{\"u}r das
Materieleistungsspektrum entwickelt, das die Materiedichtefluktuationen
statistisch bis zur zweiten Ordnung beschreibt.
Das Modell ist  rechnerisch kosteng{\"u}nstig und kann das
Materieleistungsspektrum bis auf einen Prozent Genauigkeit von
$k \sim 0.7 h^{- 1} \mathrm{Mpc}$ voraussagen.

Dar{\"u}ber hinaus untersuchten wir die Auswirkungen der Baryonen auf das
schwache Linsen Scherleistungsspektrum. Wir argumentierten, dass diese Effekte
auf kleinen Skalen $\ell \sim$ 5000 signifikant werden und bei
Vernachl{\"a}ssigung einen systematischen Fehler in die Messung und
Interpretation der kosmologischen Parameter von einigen Sigma Abweichung
einf{\"u}hrt.

Schliesslich rekonstruierten wir die Massenkarten von sechs Hubble
Frontier Field Haufen. Ihre Massenverteilung zeigt Dehnung, mehrere Kerne und
viele Substrukturen, welche einen Hinweis auf eine k\"urzliche grosse Fusion
liefern. Wir messen auch deren Klumpigkeit mit dem
Leistungsspektrum des Massenfeldes und vergleichen sie mit Simulationen welche
dem $\Lambda$CDM Model folgen.

% ----------------------------------------------------------------------
