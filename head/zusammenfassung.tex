\phantomsection
\addcontentsline{toc}{chapter}{Zusammenfassung}

%\begin{abstractslong}    %uncommenting this line, gives a different abstract heading
\chapter*{Zusammenfassung}%\begin{zusammenfassung}        %this creates the heading for the abstract page



Die Verteilung der Materie im Universum enth{\"a}lt eine F{\"u}lle von
Informationen {\"u}ber die Eigentschaften, den Energieinhalt und die zeitliche
Entwicklung des Universums. Sie kann in zwei sehr unterschiedlichen
Gr{\"o}ssenordnungen untersucht werden. Einerseits in gravitativ gebundenen
(SUB) Systemen wie zum Beispiel Galaxien und Galaxienhaufen und andererseits in
den gr{\"o}ssten  Strukturen (Large Scale Structres, LSS) des Universums.
Jedes Regime habt spezifische Anwendungen und sie gemeinsam das Verst{\"a}ndnis
f{\"u}r die Theorie der Strukturbildung und der Kosmologie.
Galaxienhaufen sind die gr{\"o}$\beta$ten gravitativ gebundenen Strukturen im
Universum, bestehen von Hunderten von Galaxien und das Gas im Inneren bewegt
sich in der Potentialmulde welche von der dunklen Materiekomponenten des Systems
erzeug wird. Wegen der Tiefe des Potentialtopfs und der hohen Gasdiche und
Temperatur kann ein solcher Haufen mit Techniken wie Gravitationslinsen,
R{\"o}ntgenbeobachtungen etc. untersucht werden und bietet eine Art kosmisches
Labor um  die Wechselwirkungen zwischen Baryonen und dunkler Materie zu
studieren oder Abweichungen der dunklen Materie vom Standartmodell untersuchen.
Die Gr{\"o}ssstrukturen andererseits werden aus anfaenglich winzigen
Dichtefluktuationen im urspr{\"u}nglichen Dichtefeld welche unter der
Gravitation instabil werden und kollabieren geformt. Das Studium der
Materieverteilung in LSS ist es m{\"o}glich, die Anfangsbedingungen und/oder
die kosmologischen Parameter einzuschr{\"a}nken.


In dieser Doktorarbeit habe ich mir mit diesen beiden Aspekten
auseinandergestetzt und erfolgreich f{\"u}nf wissenschaftliche Arbeiten
publiziert welche in dieser Doktorarbeit in gebundener Form vorliegen:

Als erstes untersuchten wir die Verteilung der Materie in drei Galaxienhaufen
und rekonstruierten ihre Massenverteilungen mit Hilfe einer parameterfreien
Formulierung des starken Gravitationslinseneffekts. In zwei dieser Haufen fanden
wir signifikante (Unterschiede) Abweichungen zwischen den Dichtemaxima und der
n{\"a}chsten Galaxie. Wir untersuchten ob diese Abweichungen einen
astrophysikalischen Ursprung haben oder Hinweise auf Selbstinteraktionen
zwischen dunkler Materie sind.

Als naechstes haben wir - einer {\"a}hnlichen Richtung folgend - die
Auswirkungen der Massenverteilung des beobachteten Haufens auf die
Zeitverzögerung zwischen verschiedenen Bildern der selben Quelle untersucht.
Wir fanden, dass in Galaxienhaufen die Zeitverz{\"o}gerung zwischen Bildern
Informationen liefert, welche verwendet werde k{\"o}nnen um die Kugelsymmetrie
des Kerns des Haufens zu beschr{\"a}nken.

In anderen Arbeiten haben wir ein analytisches Modell f{\"u}r das
Materieleistungsspektrum entwickelt, das die Materiedichtefluktuationen
statistisch (nur bis zur zweiten Ordnung) beschreibt.
Das Modell ist  rechnerisch kosteng{\"u}nstig und kann das
Materieleistungsspektrum bis auf einen Prozent Genauigkeit von
$k \sim 0.7 h^{- 1} \mathrm{Mpc}$ voraussagen.

Dar{\"u}ber hinaus untersuchten wir die Auswirkungen der Baryonen auf das
schwache Linsen Scherleistungsspektrum. Wir argumentierten, dass diese Effekte
auf kleinen Skalen $\ell \sim$ 5000 signifikant werden und bei
Vernachl{\"a}ssigung einen systematischen Fehler in die Messung und
Interpretation der kosmologischen Parameter von einigen Sigma Absweichung
einf{\"u}hrt.

Schlie$\beta$lich rekonstruieren wir die Massen Karten von sechs Hubble
Frontier Field Haufen. Ihre Massenverteilung zeigt Dehnung, mehrere Kerne und
viele Substrukturen, welche einen Hinweis auf eine kuerzlichen major merger
liefern. Wir messen auch deren Clustering Eigenschaften mit dem
Leistungsspektrum des Massefeldes und vergleichen sie Simulationen welche
dem $\Lambda$CDM Model folgen.

% ----------------------------------------------------------------------
