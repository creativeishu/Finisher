\phantomsection
\addcontentsline{toc}{chapter}{Abstract}


\chapter*{Abstract}

Distribution of matter in the Universe contains wealth of information about 
the energy content of the Universe, their properties and evolution. It can 
be studied in two very different regime. First, in individual gravitationally 
bound systems like galaxies, cluster of galaxies etc.; second, in the 
large scale structures of the Universe. Each of these regimes have specific
applications and collectively improves the understanding of the theory of 
structure formation and cosmology.  Firstly, Cluster of galaxies are
the largest gravitationally bound structures in the Universe and consists
of hundereds of galaxies and intra-cluster gas moving in the potential well 
of the large dark-matter component. Becasue of large potential well, high 
density and high temperature of the gas it can be studied with probes like 
gravitational lensing, X-ray observations etc., and provides cosmic 
laboratories to study the interactions of baryons and dark-matter or
non-standard properties of dark-matter, if any.
Secondly, the large scale structures of the Universe formed due to the evolution 
of tiny perturbations in the initial density field via gravitational 
instability and many baryonic processes. Therefore, by studying the 
distribution of matter in LSS, it is possible to constrain the initial
conditions and/or the cosmological parameters. 

In this PhD dissertation, I focussed on these two aspects and successfully 
concluded five scientific papers that are attached in this manuscript:

First, we studied the distribution of matter in three clusters of galaxies
and reconstructed their mass distribution using non-parametric technique in 
strong gravitational lensing. In some clusters, we found significant offset
between the density peaks and the nearby galaxies. We argued these offsets
could have astrophysical origin or an indication of self-interactions
of the dark-matter particles.

Second, in another project related to strong gravitational lensing, we studied
the effect of time delay, between different images of the same source,
on the mass distribution of the lensing clusters. We found that in clusters
where the steepness degeneracy is already broken by multiple background sources
at different redshift, time delay information can be used to constrain
the lopsidedness of the cluster core.

Third, we build an analytic model for the matter power spectrum that
describes the matter density fluctuations statistically (only to second 
order). The model is computationally inexpensive and predicts the
matter power spectrum to a percent level accuracy up to 
$k\sim h^{-1}\mathrm{Mpc}$. 

Fourth, we studied the effects of baryons on the sky-projected weak lensing
shear power spectrum. We argue that these effects become significant
at small scales $\ell \sim 5000$ and if ignored, it will bias the
interpretation of the cosmological parameters to many sigma.

Fifth, we reconstructed the mass maps of six Hubble Frontier Field clusters.
Their mass distribution shows elongation, multiple-cores and many sub-structures
indicating a recent major merger. We also quantified their clustering 
property with the power spectrum of the mass field and compared them
with $\Lambda$CDM simulated clusters.


% ---------------------------------------------------------------------- 
