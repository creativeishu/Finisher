\phantomsection
\addcontentsline{toc}{chapter}{Abstract}


\chapter*{Abstract}

The distribution of matter in the Universe contains a wealth of information about 
the energy content in the Universe, its properties, and evolution. It can 
be studied in two very different regimes. First, in gravitationally 
bound systems like galaxies, cluster of galaxies etc.; second, in the 
large scale structure (LSS) of the Universe. Each of these regimes have specific
applications and they collectively improve the understanding of the theory of 
structure formation and cosmology.  Firstly, clusters of galaxies are
the largest gravitationally bound structures in the Universe, consisting
of hundreds of galaxies and intra-cluster gas moving in the potential well 
of the large dark-matter component. Becasue of their deep potential well, high 
density and high temperature of their gas, the clusters can be studied with probes like 
gravitational lensing, X-ray observations etc., and provide cosmic 
laboratories to study the interactions of baryons and dark-matter, or
non-standard properties of dark-matter, if any.
Secondly, the LSS is 
formed due to the evolution 
of tiny perturbations in the initial density field via gravitational 
instability and many baryonic processes. By studying the 
distribution of matter in the LSS, it is possible to constrain the initial
conditions and/or the cosmological parameters. 

In this PhD dissertation, I focussed on these two aspects and successfully 
concluded five scientific papers, which are attached in this manuscript:

We studied the distribution of matter in three clusters of galaxies
and reconstructed their mass distribution using a non-parametric technique in 
strong gravitational lensing. In two of these clusters, we found significant offset
between the density peaks and the nearest galaxy. We discussed weather these offsets
could have an astrophysical origin or be an indication of self-interactions
of dark-matter particles.
Continuing in the same vein, we studied
the effect on time delay, between different images of the same source,
of the mass distribution of the lensing clusters. We found that in clusters
where the steepness degeneracy is already broken by multiple background sources
at different redshifts, time delay information can be used to constrain
the lopsidedness of the cluster core.

In other work, we built an analytical model for the matter power spectrum that
describes the matter density fluctuations statistically (only to second 
order). The model is computationally inexpensive and predicts the
matter power spectrum to a percent level accuracy up to 
$k\sim 0.7 h^{-1}\mathrm{Mpc}$. 

Furthermore, we studied the effects of baryons on the sky-projected weak lensing
shear power spectrum. We argued that these effects become significant
at small scales $\ell \sim 5000$ and if ignored, it will bias the
interpretation of the cosmological parameters to many sigma.

Finally, we reconstructed the mass maps of six Hubble Frontier Field clusters.
Their mass distribution shows elongation, multiple-cores, and many sub-structures
indicating a recent major merger. We also quantified their clustering 
properties with the power spectrum of the mass field and compared them
with $\Lambda$CDM simulated clusters.


% ---------------------------------------------------------------------- 
